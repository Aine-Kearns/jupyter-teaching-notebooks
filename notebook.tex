
% Default to the notebook output style

    


% Inherit from the specified cell style.




    
\documentclass[11pt]{article}

    
    
    \usepackage[T1]{fontenc}
    % Nicer default font (+ math font) than Computer Modern for most use cases
    \usepackage{mathpazo}

    % Basic figure setup, for now with no caption control since it's done
    % automatically by Pandoc (which extracts ![](path) syntax from Markdown).
    \usepackage{graphicx}
    % We will generate all images so they have a width \maxwidth. This means
    % that they will get their normal width if they fit onto the page, but
    % are scaled down if they would overflow the margins.
    \makeatletter
    \def\maxwidth{\ifdim\Gin@nat@width>\linewidth\linewidth
    \else\Gin@nat@width\fi}
    \makeatother
    \let\Oldincludegraphics\includegraphics
    % Set max figure width to be 80% of text width, for now hardcoded.
    \renewcommand{\includegraphics}[1]{\Oldincludegraphics[width=.8\maxwidth]{#1}}
    % Ensure that by default, figures have no caption (until we provide a
    % proper Figure object with a Caption API and a way to capture that
    % in the conversion process - todo).
    \usepackage{caption}
    \DeclareCaptionLabelFormat{nolabel}{}
    \captionsetup{labelformat=nolabel}

    \usepackage{adjustbox} % Used to constrain images to a maximum size 
    \usepackage{xcolor} % Allow colors to be defined
    \usepackage{enumerate} % Needed for markdown enumerations to work
    \usepackage{geometry} % Used to adjust the document margins
    \usepackage{amsmath} % Equations
    \usepackage{amssymb} % Equations
    \usepackage{textcomp} % defines textquotesingle
    % Hack from http://tex.stackexchange.com/a/47451/13684:
    \AtBeginDocument{%
        \def\PYZsq{\textquotesingle}% Upright quotes in Pygmentized code
    }
    \usepackage{upquote} % Upright quotes for verbatim code
    \usepackage{eurosym} % defines \euro
    \usepackage[mathletters]{ucs} % Extended unicode (utf-8) support
    \usepackage[utf8x]{inputenc} % Allow utf-8 characters in the tex document
    \usepackage{fancyvrb} % verbatim replacement that allows latex
    \usepackage{grffile} % extends the file name processing of package graphics 
                         % to support a larger range 
    % The hyperref package gives us a pdf with properly built
    % internal navigation ('pdf bookmarks' for the table of contents,
    % internal cross-reference links, web links for URLs, etc.)
    \usepackage{hyperref}
    \usepackage{longtable} % longtable support required by pandoc >1.10
    \usepackage{booktabs}  % table support for pandoc > 1.12.2
    \usepackage[inline]{enumitem} % IRkernel/repr support (it uses the enumerate* environment)
    \usepackage[normalem]{ulem} % ulem is needed to support strikethroughs (\sout)
                                % normalem makes italics be italics, not underlines
    

    
    
    % Colors for the hyperref package
    \definecolor{urlcolor}{rgb}{0,.145,.698}
    \definecolor{linkcolor}{rgb}{.71,0.21,0.01}
    \definecolor{citecolor}{rgb}{.12,.54,.11}

    % ANSI colors
    \definecolor{ansi-black}{HTML}{3E424D}
    \definecolor{ansi-black-intense}{HTML}{282C36}
    \definecolor{ansi-red}{HTML}{E75C58}
    \definecolor{ansi-red-intense}{HTML}{B22B31}
    \definecolor{ansi-green}{HTML}{00A250}
    \definecolor{ansi-green-intense}{HTML}{007427}
    \definecolor{ansi-yellow}{HTML}{DDB62B}
    \definecolor{ansi-yellow-intense}{HTML}{B27D12}
    \definecolor{ansi-blue}{HTML}{208FFB}
    \definecolor{ansi-blue-intense}{HTML}{0065CA}
    \definecolor{ansi-magenta}{HTML}{D160C4}
    \definecolor{ansi-magenta-intense}{HTML}{A03196}
    \definecolor{ansi-cyan}{HTML}{60C6C8}
    \definecolor{ansi-cyan-intense}{HTML}{258F8F}
    \definecolor{ansi-white}{HTML}{C5C1B4}
    \definecolor{ansi-white-intense}{HTML}{A1A6B2}

    % commands and environments needed by pandoc snippets
    % extracted from the output of `pandoc -s`
    \providecommand{\tightlist}{%
      \setlength{\itemsep}{0pt}\setlength{\parskip}{0pt}}
    \DefineVerbatimEnvironment{Highlighting}{Verbatim}{commandchars=\\\{\}}
    % Add ',fontsize=\small' for more characters per line
    \newenvironment{Shaded}{}{}
    \newcommand{\KeywordTok}[1]{\textcolor[rgb]{0.00,0.44,0.13}{\textbf{{#1}}}}
    \newcommand{\DataTypeTok}[1]{\textcolor[rgb]{0.56,0.13,0.00}{{#1}}}
    \newcommand{\DecValTok}[1]{\textcolor[rgb]{0.25,0.63,0.44}{{#1}}}
    \newcommand{\BaseNTok}[1]{\textcolor[rgb]{0.25,0.63,0.44}{{#1}}}
    \newcommand{\FloatTok}[1]{\textcolor[rgb]{0.25,0.63,0.44}{{#1}}}
    \newcommand{\CharTok}[1]{\textcolor[rgb]{0.25,0.44,0.63}{{#1}}}
    \newcommand{\StringTok}[1]{\textcolor[rgb]{0.25,0.44,0.63}{{#1}}}
    \newcommand{\CommentTok}[1]{\textcolor[rgb]{0.38,0.63,0.69}{\textit{{#1}}}}
    \newcommand{\OtherTok}[1]{\textcolor[rgb]{0.00,0.44,0.13}{{#1}}}
    \newcommand{\AlertTok}[1]{\textcolor[rgb]{1.00,0.00,0.00}{\textbf{{#1}}}}
    \newcommand{\FunctionTok}[1]{\textcolor[rgb]{0.02,0.16,0.49}{{#1}}}
    \newcommand{\RegionMarkerTok}[1]{{#1}}
    \newcommand{\ErrorTok}[1]{\textcolor[rgb]{1.00,0.00,0.00}{\textbf{{#1}}}}
    \newcommand{\NormalTok}[1]{{#1}}
    
    % Additional commands for more recent versions of Pandoc
    \newcommand{\ConstantTok}[1]{\textcolor[rgb]{0.53,0.00,0.00}{{#1}}}
    \newcommand{\SpecialCharTok}[1]{\textcolor[rgb]{0.25,0.44,0.63}{{#1}}}
    \newcommand{\VerbatimStringTok}[1]{\textcolor[rgb]{0.25,0.44,0.63}{{#1}}}
    \newcommand{\SpecialStringTok}[1]{\textcolor[rgb]{0.73,0.40,0.53}{{#1}}}
    \newcommand{\ImportTok}[1]{{#1}}
    \newcommand{\DocumentationTok}[1]{\textcolor[rgb]{0.73,0.13,0.13}{\textit{{#1}}}}
    \newcommand{\AnnotationTok}[1]{\textcolor[rgb]{0.38,0.63,0.69}{\textbf{\textit{{#1}}}}}
    \newcommand{\CommentVarTok}[1]{\textcolor[rgb]{0.38,0.63,0.69}{\textbf{\textit{{#1}}}}}
    \newcommand{\VariableTok}[1]{\textcolor[rgb]{0.10,0.09,0.49}{{#1}}}
    \newcommand{\ControlFlowTok}[1]{\textcolor[rgb]{0.00,0.44,0.13}{\textbf{{#1}}}}
    \newcommand{\OperatorTok}[1]{\textcolor[rgb]{0.40,0.40,0.40}{{#1}}}
    \newcommand{\BuiltInTok}[1]{{#1}}
    \newcommand{\ExtensionTok}[1]{{#1}}
    \newcommand{\PreprocessorTok}[1]{\textcolor[rgb]{0.74,0.48,0.00}{{#1}}}
    \newcommand{\AttributeTok}[1]{\textcolor[rgb]{0.49,0.56,0.16}{{#1}}}
    \newcommand{\InformationTok}[1]{\textcolor[rgb]{0.38,0.63,0.69}{\textbf{\textit{{#1}}}}}
    \newcommand{\WarningTok}[1]{\textcolor[rgb]{0.38,0.63,0.69}{\textbf{\textit{{#1}}}}}
    
    
    % Define a nice break command that doesn't care if a line doesn't already
    % exist.
    \def\br{\hspace*{\fill} \\* }
    % Math Jax compatability definitions
    \def\gt{>}
    \def\lt{<}
    % Document parameters
    \title{win-a-car}
    
    
    

    % Pygments definitions
    
\makeatletter
\def\PY@reset{\let\PY@it=\relax \let\PY@bf=\relax%
    \let\PY@ul=\relax \let\PY@tc=\relax%
    \let\PY@bc=\relax \let\PY@ff=\relax}
\def\PY@tok#1{\csname PY@tok@#1\endcsname}
\def\PY@toks#1+{\ifx\relax#1\empty\else%
    \PY@tok{#1}\expandafter\PY@toks\fi}
\def\PY@do#1{\PY@bc{\PY@tc{\PY@ul{%
    \PY@it{\PY@bf{\PY@ff{#1}}}}}}}
\def\PY#1#2{\PY@reset\PY@toks#1+\relax+\PY@do{#2}}

\expandafter\def\csname PY@tok@w\endcsname{\def\PY@tc##1{\textcolor[rgb]{0.73,0.73,0.73}{##1}}}
\expandafter\def\csname PY@tok@c\endcsname{\let\PY@it=\textit\def\PY@tc##1{\textcolor[rgb]{0.25,0.50,0.50}{##1}}}
\expandafter\def\csname PY@tok@cp\endcsname{\def\PY@tc##1{\textcolor[rgb]{0.74,0.48,0.00}{##1}}}
\expandafter\def\csname PY@tok@k\endcsname{\let\PY@bf=\textbf\def\PY@tc##1{\textcolor[rgb]{0.00,0.50,0.00}{##1}}}
\expandafter\def\csname PY@tok@kp\endcsname{\def\PY@tc##1{\textcolor[rgb]{0.00,0.50,0.00}{##1}}}
\expandafter\def\csname PY@tok@kt\endcsname{\def\PY@tc##1{\textcolor[rgb]{0.69,0.00,0.25}{##1}}}
\expandafter\def\csname PY@tok@o\endcsname{\def\PY@tc##1{\textcolor[rgb]{0.40,0.40,0.40}{##1}}}
\expandafter\def\csname PY@tok@ow\endcsname{\let\PY@bf=\textbf\def\PY@tc##1{\textcolor[rgb]{0.67,0.13,1.00}{##1}}}
\expandafter\def\csname PY@tok@nb\endcsname{\def\PY@tc##1{\textcolor[rgb]{0.00,0.50,0.00}{##1}}}
\expandafter\def\csname PY@tok@nf\endcsname{\def\PY@tc##1{\textcolor[rgb]{0.00,0.00,1.00}{##1}}}
\expandafter\def\csname PY@tok@nc\endcsname{\let\PY@bf=\textbf\def\PY@tc##1{\textcolor[rgb]{0.00,0.00,1.00}{##1}}}
\expandafter\def\csname PY@tok@nn\endcsname{\let\PY@bf=\textbf\def\PY@tc##1{\textcolor[rgb]{0.00,0.00,1.00}{##1}}}
\expandafter\def\csname PY@tok@ne\endcsname{\let\PY@bf=\textbf\def\PY@tc##1{\textcolor[rgb]{0.82,0.25,0.23}{##1}}}
\expandafter\def\csname PY@tok@nv\endcsname{\def\PY@tc##1{\textcolor[rgb]{0.10,0.09,0.49}{##1}}}
\expandafter\def\csname PY@tok@no\endcsname{\def\PY@tc##1{\textcolor[rgb]{0.53,0.00,0.00}{##1}}}
\expandafter\def\csname PY@tok@nl\endcsname{\def\PY@tc##1{\textcolor[rgb]{0.63,0.63,0.00}{##1}}}
\expandafter\def\csname PY@tok@ni\endcsname{\let\PY@bf=\textbf\def\PY@tc##1{\textcolor[rgb]{0.60,0.60,0.60}{##1}}}
\expandafter\def\csname PY@tok@na\endcsname{\def\PY@tc##1{\textcolor[rgb]{0.49,0.56,0.16}{##1}}}
\expandafter\def\csname PY@tok@nt\endcsname{\let\PY@bf=\textbf\def\PY@tc##1{\textcolor[rgb]{0.00,0.50,0.00}{##1}}}
\expandafter\def\csname PY@tok@nd\endcsname{\def\PY@tc##1{\textcolor[rgb]{0.67,0.13,1.00}{##1}}}
\expandafter\def\csname PY@tok@s\endcsname{\def\PY@tc##1{\textcolor[rgb]{0.73,0.13,0.13}{##1}}}
\expandafter\def\csname PY@tok@sd\endcsname{\let\PY@it=\textit\def\PY@tc##1{\textcolor[rgb]{0.73,0.13,0.13}{##1}}}
\expandafter\def\csname PY@tok@si\endcsname{\let\PY@bf=\textbf\def\PY@tc##1{\textcolor[rgb]{0.73,0.40,0.53}{##1}}}
\expandafter\def\csname PY@tok@se\endcsname{\let\PY@bf=\textbf\def\PY@tc##1{\textcolor[rgb]{0.73,0.40,0.13}{##1}}}
\expandafter\def\csname PY@tok@sr\endcsname{\def\PY@tc##1{\textcolor[rgb]{0.73,0.40,0.53}{##1}}}
\expandafter\def\csname PY@tok@ss\endcsname{\def\PY@tc##1{\textcolor[rgb]{0.10,0.09,0.49}{##1}}}
\expandafter\def\csname PY@tok@sx\endcsname{\def\PY@tc##1{\textcolor[rgb]{0.00,0.50,0.00}{##1}}}
\expandafter\def\csname PY@tok@m\endcsname{\def\PY@tc##1{\textcolor[rgb]{0.40,0.40,0.40}{##1}}}
\expandafter\def\csname PY@tok@gh\endcsname{\let\PY@bf=\textbf\def\PY@tc##1{\textcolor[rgb]{0.00,0.00,0.50}{##1}}}
\expandafter\def\csname PY@tok@gu\endcsname{\let\PY@bf=\textbf\def\PY@tc##1{\textcolor[rgb]{0.50,0.00,0.50}{##1}}}
\expandafter\def\csname PY@tok@gd\endcsname{\def\PY@tc##1{\textcolor[rgb]{0.63,0.00,0.00}{##1}}}
\expandafter\def\csname PY@tok@gi\endcsname{\def\PY@tc##1{\textcolor[rgb]{0.00,0.63,0.00}{##1}}}
\expandafter\def\csname PY@tok@gr\endcsname{\def\PY@tc##1{\textcolor[rgb]{1.00,0.00,0.00}{##1}}}
\expandafter\def\csname PY@tok@ge\endcsname{\let\PY@it=\textit}
\expandafter\def\csname PY@tok@gs\endcsname{\let\PY@bf=\textbf}
\expandafter\def\csname PY@tok@gp\endcsname{\let\PY@bf=\textbf\def\PY@tc##1{\textcolor[rgb]{0.00,0.00,0.50}{##1}}}
\expandafter\def\csname PY@tok@go\endcsname{\def\PY@tc##1{\textcolor[rgb]{0.53,0.53,0.53}{##1}}}
\expandafter\def\csname PY@tok@gt\endcsname{\def\PY@tc##1{\textcolor[rgb]{0.00,0.27,0.87}{##1}}}
\expandafter\def\csname PY@tok@err\endcsname{\def\PY@bc##1{\setlength{\fboxsep}{0pt}\fcolorbox[rgb]{1.00,0.00,0.00}{1,1,1}{\strut ##1}}}
\expandafter\def\csname PY@tok@kc\endcsname{\let\PY@bf=\textbf\def\PY@tc##1{\textcolor[rgb]{0.00,0.50,0.00}{##1}}}
\expandafter\def\csname PY@tok@kd\endcsname{\let\PY@bf=\textbf\def\PY@tc##1{\textcolor[rgb]{0.00,0.50,0.00}{##1}}}
\expandafter\def\csname PY@tok@kn\endcsname{\let\PY@bf=\textbf\def\PY@tc##1{\textcolor[rgb]{0.00,0.50,0.00}{##1}}}
\expandafter\def\csname PY@tok@kr\endcsname{\let\PY@bf=\textbf\def\PY@tc##1{\textcolor[rgb]{0.00,0.50,0.00}{##1}}}
\expandafter\def\csname PY@tok@bp\endcsname{\def\PY@tc##1{\textcolor[rgb]{0.00,0.50,0.00}{##1}}}
\expandafter\def\csname PY@tok@fm\endcsname{\def\PY@tc##1{\textcolor[rgb]{0.00,0.00,1.00}{##1}}}
\expandafter\def\csname PY@tok@vc\endcsname{\def\PY@tc##1{\textcolor[rgb]{0.10,0.09,0.49}{##1}}}
\expandafter\def\csname PY@tok@vg\endcsname{\def\PY@tc##1{\textcolor[rgb]{0.10,0.09,0.49}{##1}}}
\expandafter\def\csname PY@tok@vi\endcsname{\def\PY@tc##1{\textcolor[rgb]{0.10,0.09,0.49}{##1}}}
\expandafter\def\csname PY@tok@vm\endcsname{\def\PY@tc##1{\textcolor[rgb]{0.10,0.09,0.49}{##1}}}
\expandafter\def\csname PY@tok@sa\endcsname{\def\PY@tc##1{\textcolor[rgb]{0.73,0.13,0.13}{##1}}}
\expandafter\def\csname PY@tok@sb\endcsname{\def\PY@tc##1{\textcolor[rgb]{0.73,0.13,0.13}{##1}}}
\expandafter\def\csname PY@tok@sc\endcsname{\def\PY@tc##1{\textcolor[rgb]{0.73,0.13,0.13}{##1}}}
\expandafter\def\csname PY@tok@dl\endcsname{\def\PY@tc##1{\textcolor[rgb]{0.73,0.13,0.13}{##1}}}
\expandafter\def\csname PY@tok@s2\endcsname{\def\PY@tc##1{\textcolor[rgb]{0.73,0.13,0.13}{##1}}}
\expandafter\def\csname PY@tok@sh\endcsname{\def\PY@tc##1{\textcolor[rgb]{0.73,0.13,0.13}{##1}}}
\expandafter\def\csname PY@tok@s1\endcsname{\def\PY@tc##1{\textcolor[rgb]{0.73,0.13,0.13}{##1}}}
\expandafter\def\csname PY@tok@mb\endcsname{\def\PY@tc##1{\textcolor[rgb]{0.40,0.40,0.40}{##1}}}
\expandafter\def\csname PY@tok@mf\endcsname{\def\PY@tc##1{\textcolor[rgb]{0.40,0.40,0.40}{##1}}}
\expandafter\def\csname PY@tok@mh\endcsname{\def\PY@tc##1{\textcolor[rgb]{0.40,0.40,0.40}{##1}}}
\expandafter\def\csname PY@tok@mi\endcsname{\def\PY@tc##1{\textcolor[rgb]{0.40,0.40,0.40}{##1}}}
\expandafter\def\csname PY@tok@il\endcsname{\def\PY@tc##1{\textcolor[rgb]{0.40,0.40,0.40}{##1}}}
\expandafter\def\csname PY@tok@mo\endcsname{\def\PY@tc##1{\textcolor[rgb]{0.40,0.40,0.40}{##1}}}
\expandafter\def\csname PY@tok@ch\endcsname{\let\PY@it=\textit\def\PY@tc##1{\textcolor[rgb]{0.25,0.50,0.50}{##1}}}
\expandafter\def\csname PY@tok@cm\endcsname{\let\PY@it=\textit\def\PY@tc##1{\textcolor[rgb]{0.25,0.50,0.50}{##1}}}
\expandafter\def\csname PY@tok@cpf\endcsname{\let\PY@it=\textit\def\PY@tc##1{\textcolor[rgb]{0.25,0.50,0.50}{##1}}}
\expandafter\def\csname PY@tok@c1\endcsname{\let\PY@it=\textit\def\PY@tc##1{\textcolor[rgb]{0.25,0.50,0.50}{##1}}}
\expandafter\def\csname PY@tok@cs\endcsname{\let\PY@it=\textit\def\PY@tc##1{\textcolor[rgb]{0.25,0.50,0.50}{##1}}}

\def\PYZbs{\char`\\}
\def\PYZus{\char`\_}
\def\PYZob{\char`\{}
\def\PYZcb{\char`\}}
\def\PYZca{\char`\^}
\def\PYZam{\char`\&}
\def\PYZlt{\char`\<}
\def\PYZgt{\char`\>}
\def\PYZsh{\char`\#}
\def\PYZpc{\char`\%}
\def\PYZdl{\char`\$}
\def\PYZhy{\char`\-}
\def\PYZsq{\char`\'}
\def\PYZdq{\char`\"}
\def\PYZti{\char`\~}
% for compatibility with earlier versions
\def\PYZat{@}
\def\PYZlb{[}
\def\PYZrb{]}
\makeatother


    % Exact colors from NB
    \definecolor{incolor}{rgb}{0.0, 0.0, 0.5}
    \definecolor{outcolor}{rgb}{0.545, 0.0, 0.0}



    
    % Prevent overflowing lines due to hard-to-break entities
    \sloppy 
    % Setup hyperref package
    \hypersetup{
      breaklinks=true,  % so long urls are correctly broken across lines
      colorlinks=true,
      urlcolor=urlcolor,
      linkcolor=linkcolor,
      citecolor=citecolor,
      }
    % Slightly bigger margins than the latex defaults
    
    \geometry{verbose,tmargin=1in,bmargin=1in,lmargin=1in,rmargin=1in}
    
    

    \begin{document}
    
    
    \maketitle
    
    

    
    \hypertarget{win-a-car}{%
\section{Win a car}\label{win-a-car}}

The purpose of this Jupyter notebook is twofold. First, you'll play a
simple game in which you are asked to make two choices. At the end,
you'll be asked to submit the outcome of the game. The outcome will be
combined with the outcomes of your classmates and analysed in class.
Second, the sheet will be a quick (re-)introduction to Jupyter notebooks
and Python.

This is just a bit of fun - you can't actually win a car. Please don't
Google the game before you get through the whole notebook and submit
your game. There's no good reason to, remember this is just for fun. If
you do Google anything, you'll just end up skewing the overall results
of the class. Please just play the game the way you feel it.

\begin{center}\rule{0.5\linewidth}{\linethickness}\end{center}

    \hypertarget{background}{%
\subsection{Background}\label{background}}

The game we're going to play is adapted from an old TV show. The
contestant, after going through a number of rounds, is given the
opportunity to win a car. They're presented with three closed doors: A,
B and C.

They're told that a car has been randomly placed behind one of these
doors. The other two doors each have a goat behind them. All the
contestant has to do is pick a door to open. There's no extra
information and there's no puzzle to solve - it's a straight-forward
game of chance.

There are a couple of things I'd like to know from you before you play
the game. ***

    \hypertarget{running-this-notebook}{%
\subsection{Running this notebook}\label{running-this-notebook}}

This is a Jupyter notebook. You should download it and run it through
Jupyter on your own machine.

\begin{verbatim}
$ jupyter notebook
\end{verbatim}

The notebook should be run in one session from top to bottom. If you
mess up at any stage, you can click \texttt{Restart\ \&\ Clear\ Output}
from the Kernel menu at the top.

\begin{center}\rule{0.5\linewidth}{\linethickness}\end{center}

    \hypertarget{starting-variables}{%
\subsection{Starting variables}\label{starting-variables}}

To begin, I'd like you to set each of the following four variables to
either True or False in the cell below. Remember, please don't Google
anything for now - just use the information currently in your head.

Set \texttt{einstein} to True if you have ever heard of someone called
Albert Einstein, False otherwise. Set \texttt{monty} to True if you have
ever heard of someone called Monty Hall, False otherwise. Set
\texttt{savant} to True if you have ever heard of someone called Marilyn
vos Savant, False otherwise. Set \texttt{game} to True if you have ever
heard of this game (involving cars and goats or otherwise) before, False
otherwise.

So, if you know who Einstein is but not Monty Hall or Marilyn vos Savant
and you've never heard of a game with goats behind doors before then
type the following in the cell below and press Shift+Enter.

\begin{Shaded}
\begin{Highlighting}[]
\NormalTok{einstein }\OperatorTok{=} \VariableTok{True}
\NormalTok{monty }\OperatorTok{=} \VariableTok{False}
\NormalTok{savant }\OperatorTok{=} \VariableTok{False}
\NormalTok{game }\OperatorTok{=} \VariableTok{False}
\end{Highlighting}
\end{Shaded}

    \begin{Verbatim}[commandchars=\\\{\}]
{\color{incolor}In [{\color{incolor} }]:} \PY{c+c1}{\PYZsh{} An example of setting the variables described above.}
        \PY{n}{einstein} \PY{o}{=} 
        \PY{n}{monty} \PY{o}{=} 
        \PY{n}{savant} \PY{o}{=} 
        \PY{n}{game} \PY{o}{=} 
\end{Verbatim}


    \begin{center}\rule{0.5\linewidth}{\linethickness}\end{center}

\hypertarget{quick-check-1}{%
\subsection{Quick check 1}\label{quick-check-1}}

Run the cell below to make sure everything at this point is running
okay.

    \begin{Verbatim}[commandchars=\\\{\}]
{\color{incolor}In [{\color{incolor} }]:} \PY{c+c1}{\PYZsh{} Assume everything is running okay.}
        \PY{n}{aok} \PY{o}{=} \PY{k+kc}{True}
        
        \PY{c+c1}{\PYZsh{} Make sure the einstein variable is set to True or False.}
        \PY{k}{try}\PY{p}{:}
            \PY{k}{if} \PY{n}{einstein} \PY{o+ow}{not} \PY{o+ow}{in} \PY{p}{[}\PY{k+kc}{True}\PY{p}{,} \PY{k+kc}{False}\PY{p}{]}\PY{p}{:}
                \PY{n+nb}{print}\PY{p}{(}\PY{l+s+s2}{\PYZdq{}}\PY{l+s+s2}{Error: please set the einstein variable to either True or False!}\PY{l+s+s2}{\PYZdq{}}\PY{p}{)}
                \PY{n}{aok} \PY{o}{=} \PY{k+kc}{False}
        \PY{k}{except} \PY{n+ne}{NameError}\PY{p}{:}
            \PY{n+nb}{print}\PY{p}{(}\PY{l+s+s2}{\PYZdq{}}\PY{l+s+s2}{Error: please set the einstein variable to either True or False!}\PY{l+s+s2}{\PYZdq{}}\PY{p}{)}
            \PY{n}{aok} \PY{o}{=} \PY{k+kc}{False}
        
        \PY{c+c1}{\PYZsh{} Make sure the monty variable is set to True or False.}
        \PY{k}{try}\PY{p}{:}
            \PY{k}{if} \PY{n}{monty} \PY{o+ow}{not} \PY{o+ow}{in} \PY{p}{[}\PY{k+kc}{True}\PY{p}{,} \PY{k+kc}{False}\PY{p}{]}\PY{p}{:}
                \PY{n+nb}{print}\PY{p}{(}\PY{l+s+s2}{\PYZdq{}}\PY{l+s+s2}{Error: please set the monty variable to either True or False!}\PY{l+s+s2}{\PYZdq{}}\PY{p}{)}
                \PY{n}{aok} \PY{o}{=} \PY{k+kc}{False}
        \PY{k}{except} \PY{n+ne}{NameError}\PY{p}{:}
            \PY{n+nb}{print}\PY{p}{(}\PY{l+s+s2}{\PYZdq{}}\PY{l+s+s2}{Error: please set the monty variable to either True or False!}\PY{l+s+s2}{\PYZdq{}}\PY{p}{)}
            \PY{n}{aok} \PY{o}{=} \PY{k+kc}{False}
        
        \PY{c+c1}{\PYZsh{} Make sure the savant variable is set to True or False.}
        \PY{k}{try}\PY{p}{:}
            \PY{k}{if} \PY{n}{savant} \PY{o+ow}{not} \PY{o+ow}{in} \PY{p}{[}\PY{k+kc}{True}\PY{p}{,} \PY{k+kc}{False}\PY{p}{]}\PY{p}{:}
                \PY{n+nb}{print}\PY{p}{(}\PY{l+s+s2}{\PYZdq{}}\PY{l+s+s2}{Error: please set the savant variable to either True or False!}\PY{l+s+s2}{\PYZdq{}}\PY{p}{)}
                \PY{n}{aok} \PY{o}{=} \PY{k+kc}{False}
        \PY{k}{except} \PY{n+ne}{NameError}\PY{p}{:}
            \PY{n+nb}{print}\PY{p}{(}\PY{l+s+s2}{\PYZdq{}}\PY{l+s+s2}{Error: please set the savant variable to either True or False!}\PY{l+s+s2}{\PYZdq{}}\PY{p}{)}
            \PY{n}{aok} \PY{o}{=} \PY{k+kc}{False}
            
        \PY{c+c1}{\PYZsh{} Make sure the game variable is set to True or False.}
        \PY{k}{try}\PY{p}{:}
            \PY{k}{if} \PY{n}{game} \PY{o+ow}{not} \PY{o+ow}{in} \PY{p}{[}\PY{k+kc}{True}\PY{p}{,} \PY{k+kc}{False}\PY{p}{]}\PY{p}{:}
                \PY{n+nb}{print}\PY{p}{(}\PY{l+s+s2}{\PYZdq{}}\PY{l+s+s2}{Error: please set the game variable to either True or False!}\PY{l+s+s2}{\PYZdq{}}\PY{p}{)}
                \PY{n}{aok} \PY{o}{=} \PY{k+kc}{False}
        \PY{k}{except} \PY{n+ne}{NameError}\PY{p}{:}
            \PY{n+nb}{print}\PY{p}{(}\PY{l+s+s2}{\PYZdq{}}\PY{l+s+s2}{Error: please set the game variable to either True or False!}\PY{l+s+s2}{\PYZdq{}}\PY{p}{)}
            \PY{n}{aok} \PY{o}{=} \PY{k+kc}{False}
        
        \PY{c+c1}{\PYZsh{} If there are no problems, print a statement to continue.}
        \PY{k}{if} \PY{n}{aok}\PY{p}{:}
            \PY{n+nb}{print}\PY{p}{(}\PY{l+s+s2}{\PYZdq{}}\PY{l+s+s2}{Quick Check 1: everything is running okay, please continue.}\PY{l+s+s2}{\PYZdq{}}\PY{p}{)}
\end{Verbatim}


    \begin{center}\rule{0.5\linewidth}{\linethickness}\end{center}

\hypertarget{beginning-the-game}{%
\subsection{Beginning the game}\label{beginning-the-game}}

    

    Below is some Python code to generate a random choice between three
doors as pictured above: yellow, red and blue. The code will randomly
choose one of the doors to put the car behind. There's no cheating here
- the car is as likely to be behind any one of the doors.

The colour of the door with the car is stored as a string in the
variable \texttt{car}. The yellow door is indicated by the string
\texttt{\textquotesingle{}yellow\textquotesingle{}}, the red by the
string \texttt{\textquotesingle{}red\textquotesingle{}} and the blue by
\texttt{\textquotesingle{}blue\textquotesingle{}}. Please don't look at
the value of the \texttt{car} variable, at least until the game is over.
If you re-run the cell you will again randomly select a door, possibly
the same, possibly different.

    \begin{Verbatim}[commandchars=\\\{\}]
{\color{incolor}In [{\color{incolor} }]:} \PY{c+c1}{\PYZsh{} Python provides a library called random to generate pseudo\PYZhy{}random numbers and do stuff with them.}
        \PY{k+kn}{import} \PY{n+nn}{random}
        
        \PY{c+c1}{\PYZsh{} The three doors in a list.}
        \PY{n}{doors} \PY{o}{=} \PY{p}{[}\PY{l+s+s1}{\PYZsq{}}\PY{l+s+s1}{yellow}\PY{l+s+s1}{\PYZsq{}}\PY{p}{,} \PY{l+s+s1}{\PYZsq{}}\PY{l+s+s1}{red}\PY{l+s+s1}{\PYZsq{}}\PY{p}{,} \PY{l+s+s1}{\PYZsq{}}\PY{l+s+s1}{blue}\PY{l+s+s1}{\PYZsq{}}\PY{p}{]}
        
        \PY{c+c1}{\PYZsh{} Pick a random door.}
        \PY{n}{car} \PY{o}{=} \PY{n}{random}\PY{o}{.}\PY{n}{choice}\PY{p}{(}\PY{n}{doors}\PY{p}{)}
        
        \PY{c+c1}{\PYZsh{} Tell the user we\PYZsq{}ve picked.}
        \PY{n+nb}{print}\PY{p}{(}\PY{l+s+s2}{\PYZdq{}}\PY{l+s+s2}{From Python: I}\PY{l+s+s2}{\PYZsq{}}\PY{l+s+s2}{ve randomly selected a door to put the car behind.}\PY{l+s+s2}{\PYZdq{}}\PY{p}{)}
\end{Verbatim}


    \hypertarget{your-first-choice}{%
\subsection{Your first choice}\label{your-first-choice}}

Okay, so here's your first of two choices as the contestant: pick a door
that you think the car might be behind.

To indicate your preference, in the cell below set the variable
\texttt{pick} to \texttt{\textquotesingle{}yellow\textquotesingle{}} if
you want to pick the yellow door,
\texttt{\textquotesingle{}red\textquotesingle{}} for the red door or
\texttt{\textquotesingle{}blue\textquotesingle{}} for the blue.

Just to be clear, in the cell above Python randomly selected a door to
put the car behind (implying there are goats behind the other two
doors.) You shouldn't have looked at what Python chose - it's stored in
the \texttt{car} variable. Now you're being asked to guess what door
Python might have chosen, based on no information really - it's just a
guess. You should store that guess in the \texttt{pick} variable.

    \begin{Verbatim}[commandchars=\\\{\}]
{\color{incolor}In [{\color{incolor} }]:} \PY{c+c1}{\PYZsh{} An example of picking the blue door.}
        \PY{n}{pick} \PY{o}{=} 
\end{Verbatim}


    \begin{center}\rule{0.5\linewidth}{\linethickness}\end{center}

\hypertarget{quick-check-2}{%
\subsection{Quick check 2}\label{quick-check-2}}

Run the cell below to make sure everything at this point is running
okay.

    \begin{Verbatim}[commandchars=\\\{\}]
{\color{incolor}In [{\color{incolor} }]:} \PY{c+c1}{\PYZsh{} Assume everything is running okay.}
        \PY{n}{aok} \PY{o}{=} \PY{k+kc}{True}
        
        \PY{c+c1}{\PYZsh{} Make sure car is set to a door.}
        \PY{k}{try}\PY{p}{:}
            \PY{k}{if} \PY{n}{car} \PY{o+ow}{not} \PY{o+ow}{in} \PY{n}{doors}\PY{p}{:}
                \PY{n+nb}{print}\PY{p}{(}\PY{l+s+s2}{\PYZdq{}}\PY{l+s+s2}{Error: please set the car variable to one of }\PY{l+s+s2}{\PYZsq{}}\PY{l+s+s2}{Y}\PY{l+s+s2}{\PYZsq{}}\PY{l+s+s2}{, }\PY{l+s+s2}{\PYZsq{}}\PY{l+s+s2}{R}\PY{l+s+s2}{\PYZsq{}}\PY{l+s+s2}{ or }\PY{l+s+s2}{\PYZsq{}}\PY{l+s+s2}{B}\PY{l+s+s2}{\PYZsq{}}\PY{l+s+s2}{!}\PY{l+s+s2}{\PYZdq{}}\PY{p}{)}
                \PY{n}{aok} \PY{o}{=} \PY{k+kc}{False}
        \PY{k}{except} \PY{n+ne}{NameError}\PY{p}{:}
            \PY{n+nb}{print}\PY{p}{(}\PY{l+s+s2}{\PYZdq{}}\PY{l+s+s2}{Error: please set the car variable to one of }\PY{l+s+s2}{\PYZsq{}}\PY{l+s+s2}{Y}\PY{l+s+s2}{\PYZsq{}}\PY{l+s+s2}{, }\PY{l+s+s2}{\PYZsq{}}\PY{l+s+s2}{R}\PY{l+s+s2}{\PYZsq{}}\PY{l+s+s2}{ or }\PY{l+s+s2}{\PYZsq{}}\PY{l+s+s2}{B}\PY{l+s+s2}{\PYZsq{}}\PY{l+s+s2}{!}\PY{l+s+s2}{\PYZdq{}}\PY{p}{)}
            \PY{n}{aok} \PY{o}{=} \PY{k+kc}{False}
        
        \PY{c+c1}{\PYZsh{} Make sure pick is set to a door.}
        \PY{k}{try}\PY{p}{:}
            \PY{k}{if} \PY{n}{pick} \PY{o+ow}{not} \PY{o+ow}{in} \PY{n}{doors}\PY{p}{:}
                \PY{n+nb}{print}\PY{p}{(}\PY{l+s+s2}{\PYZdq{}}\PY{l+s+s2}{Error: please set the pick variable to one of }\PY{l+s+s2}{\PYZsq{}}\PY{l+s+s2}{Y}\PY{l+s+s2}{\PYZsq{}}\PY{l+s+s2}{, }\PY{l+s+s2}{\PYZsq{}}\PY{l+s+s2}{R}\PY{l+s+s2}{\PYZsq{}}\PY{l+s+s2}{ or }\PY{l+s+s2}{\PYZsq{}}\PY{l+s+s2}{B}\PY{l+s+s2}{\PYZsq{}}\PY{l+s+s2}{!}\PY{l+s+s2}{\PYZdq{}}\PY{p}{)}
                \PY{n}{aok} \PY{o}{=} \PY{k+kc}{False}
        \PY{k}{except} \PY{n+ne}{NameError}\PY{p}{:}
            \PY{n+nb}{print}\PY{p}{(}\PY{l+s+s2}{\PYZdq{}}\PY{l+s+s2}{Error: please set the pick variable to one of }\PY{l+s+s2}{\PYZsq{}}\PY{l+s+s2}{Y}\PY{l+s+s2}{\PYZsq{}}\PY{l+s+s2}{, }\PY{l+s+s2}{\PYZsq{}}\PY{l+s+s2}{R}\PY{l+s+s2}{\PYZsq{}}\PY{l+s+s2}{ or }\PY{l+s+s2}{\PYZsq{}}\PY{l+s+s2}{B}\PY{l+s+s2}{\PYZsq{}}\PY{l+s+s2}{!}\PY{l+s+s2}{\PYZdq{}}\PY{p}{)}
            \PY{n}{aok} \PY{o}{=} \PY{k+kc}{False}
        
        \PY{c+c1}{\PYZsh{} If there are no problems, print a statement to continue.}
        \PY{k}{if} \PY{n}{aok}\PY{p}{:}
            \PY{n+nb}{print}\PY{p}{(}\PY{l+s+s2}{\PYZdq{}}\PY{l+s+s2}{Quick Check 2: everything is running okay, please continue.}\PY{l+s+s2}{\PYZdq{}}\PY{p}{)}
\end{Verbatim}


    \begin{center}\rule{0.5\linewidth}{\linethickness}\end{center}

    \hypertarget{twist}{%
\subsection{Twist}\label{twist}}

At this point, the car has been randomly placed behind one of the doors
and you've selected the door you think it might be behind. Next the host
throws in a twist: they open one of the doors that you haven't picked
and shows you there is a goat behind it. They then ask you whether you
want to stick with your current pick or do you want to change to the
other door that they didn't open. Run the cell below to simulate the
host doing this.

    \begin{Verbatim}[commandchars=\\\{\}]
{\color{incolor}In [{\color{incolor} }]:} \PY{c+c1}{\PYZsh{} Pick a random door that was neither picked by the contestant nor has the car behind it.}
        \PY{n}{show} \PY{o}{=} \PY{n}{random}\PY{o}{.}\PY{n}{choice}\PY{p}{(}\PY{p}{[}\PY{n}{door} \PY{k}{for} \PY{n}{door} \PY{o+ow}{in} \PY{n}{doors} \PY{k}{if} \PY{n}{door} \PY{o}{!=} \PY{n}{car} \PY{o+ow}{and} \PY{n}{door} \PY{o}{!=} \PY{n}{pick}\PY{p}{]}\PY{p}{)}
        
        \PY{c+c1}{\PYZsh{} Figure out which door was not opened or picked.}
        \PY{n}{notopen} \PY{o}{=} \PY{p}{[}\PY{n}{door} \PY{k}{for} \PY{n}{door} \PY{o+ow}{in} \PY{n}{doors} \PY{k}{if} \PY{n}{door} \PY{o}{!=} \PY{n}{pick} \PY{o+ow}{and} \PY{n}{door} \PY{o}{!=} \PY{n}{show}\PY{p}{]}\PY{p}{[}\PY{l+m+mi}{0}\PY{p}{]}
        
        \PY{c+c1}{\PYZsh{} Tell the user which door was opened by the host.}
        \PY{k+kn}{from} \PY{n+nn}{IPython}\PY{n+nn}{.}\PY{n+nn}{display} \PY{k}{import} \PY{n}{Markdown}\PY{p}{,} \PY{n}{display}
        \PY{n}{display}\PY{p}{(}\PY{n}{Markdown}\PY{p}{(}\PY{n}{f}\PY{l+s+s2}{\PYZdq{}\PYZdq{}\PYZdq{}}
        \PY{l+s+s2}{\PYZsh{}\PYZsh{}\PYZsh{}\PYZsh{} Host:}
        \PY{l+s+s2}{\PYZgt{} *You chose the }\PY{l+s+si}{\PYZob{}pick\PYZcb{}}\PY{l+s+s2}{ door. The producers have told me I can reveal to you that there is a goat behind the }\PY{l+s+si}{\PYZob{}show\PYZcb{}}\PY{l+s+s2}{ door.}
        \PY{l+s+s2}{My question to you now is this: do you want to stick with the }\PY{l+s+si}{\PYZob{}pick\PYZcb{}}\PY{l+s+s2}{ door or do you want to switch to the }\PY{l+s+si}{\PYZob{}notopen\PYZcb{}}\PY{l+s+s2}{ door?*}
        \PY{l+s+s2}{\PYZdq{}\PYZdq{}\PYZdq{}}\PY{p}{)}\PY{p}{)}
\end{Verbatim}


    \begin{center}\rule{0.5\linewidth}{\linethickness}\end{center}

\hypertarget{your-second-choice}{%
\subsection{Your second choice}\label{your-second-choice}}

So, you have the choice now between staying with your original choice of
door or switching to the other door. Once you make your decision, the
doors will be opened to reveal which one has the car behind it.

In the next cell, indicate whether you will stay with your current
choice of door or switch by setting the \texttt{stay} variable to True
or False. Set \texttt{stay} to True if you would like to stick with your
current choice of door. Set \texttt{stay} to False if you would like to
switch to the other door offered by the host.

    \begin{Verbatim}[commandchars=\\\{\}]
{\color{incolor}In [{\color{incolor} }]:} \PY{c+c1}{\PYZsh{} An example of sticking with your current choice.}
        \PY{n}{stay} \PY{o}{=} 
\end{Verbatim}


    \begin{center}\rule{0.5\linewidth}{\linethickness}\end{center}

\hypertarget{quick-check-3}{%
\subsection{Quick check 3}\label{quick-check-3}}

Run the cell below to make sure everything at this point is running
okay.

    \begin{Verbatim}[commandchars=\\\{\}]
{\color{incolor}In [{\color{incolor} }]:} \PY{c+c1}{\PYZsh{} Assume everything is running okay.}
        \PY{n}{aok} \PY{o}{=} \PY{k+kc}{True}
        
        \PY{c+c1}{\PYZsh{} Make sure the stay variable is set to True or False.}
        \PY{k}{try}\PY{p}{:}
            \PY{k}{if} \PY{n}{stay} \PY{o+ow}{not} \PY{o+ow}{in} \PY{p}{[}\PY{k+kc}{True}\PY{p}{,} \PY{k+kc}{False}\PY{p}{]}\PY{p}{:}
                \PY{n+nb}{print}\PY{p}{(}\PY{l+s+s2}{\PYZdq{}}\PY{l+s+s2}{Error: please set the stay variable to either True or False!}\PY{l+s+s2}{\PYZdq{}}\PY{p}{)}
                \PY{n}{aok} \PY{o}{=} \PY{k+kc}{False}
        \PY{k}{except} \PY{n+ne}{NameError}\PY{p}{:}
            \PY{n+nb}{print}\PY{p}{(}\PY{l+s+s2}{\PYZdq{}}\PY{l+s+s2}{Error: please set the stay variable to either True or False!}\PY{l+s+s2}{\PYZdq{}}\PY{p}{)}
            \PY{n}{aok} \PY{o}{=} \PY{k+kc}{False}
        
        \PY{c+c1}{\PYZsh{} If there are no problems, print a statement to continue.}
        \PY{k}{if} \PY{n}{aok}\PY{p}{:}
            \PY{n+nb}{print}\PY{p}{(}\PY{l+s+s2}{\PYZdq{}}\PY{l+s+s2}{Quick Check 3: everything is running okay, please continue.}\PY{l+s+s2}{\PYZdq{}}\PY{p}{)}
\end{Verbatim}


    \begin{center}\rule{0.5\linewidth}{\linethickness}\end{center}

\hypertarget{the-reveal}{%
\subsection{The reveal}\label{the-reveal}}

Run the cell below to see if you would have won the car. Remember it's
just a bit of fun so it doesn't matter whether or not you won.

    \begin{Verbatim}[commandchars=\\\{\}]
{\color{incolor}In [{\color{incolor} }]:} \PY{c+c1}{\PYZsh{} Tell the user which door was opened by the host.}
        \PY{k+kn}{from} \PY{n+nn}{IPython}\PY{n+nn}{.}\PY{n+nn}{display} \PY{k}{import} \PY{n}{Markdown}\PY{p}{,} \PY{n}{display}
        \PY{n}{display}\PY{p}{(}\PY{n}{Markdown}\PY{p}{(}\PY{n}{f}\PY{l+s+s2}{\PYZdq{}\PYZdq{}\PYZdq{}}
        \PY{l+s+s2}{\PYZsh{}\PYZsh{}\PYZsh{}\PYZsh{} Host:}
        \PY{l+s+s2}{\PYZgt{} *You originally picked the }\PY{l+s+si}{\PYZob{}pick\PYZcb{}}\PY{l+s+s2}{ door and I showed you there was a goat behind the }\PY{l+s+si}{\PYZob{}show\PYZcb{}}\PY{l+s+s2}{ door.*}
        \PY{l+s+s2}{\PYZgt{}}
        \PY{l+s+s2}{\PYZgt{} *I asked you if you wanted to stay with the }\PY{l+s+si}{\PYZob{}pick\PYZcb{}}\PY{l+s+s2}{ door or change to the }\PY{l+s+si}{\PYZob{}notopen\PYZcb{}}\PY{l+s+s2}{ door. You decided to }\PY{l+s+s2}{\PYZob{}}\PY{l+s+s2}{\PYZsq{}}\PY{l+s+s2}{stay}\PY{l+s+s2}{\PYZsq{}}\PY{l+s+s2}{ if stay else }\PY{l+s+s2}{\PYZsq{}}\PY{l+s+s2}{change}\PY{l+s+s2}{\PYZsq{}}\PY{l+s+s2}{\PYZcb{}.*}
        \PY{l+s+s2}{\PYZgt{}}
        \PY{l+s+s2}{\PYZgt{} *I can now reveal to you that the car was behind the }\PY{l+s+si}{\PYZob{}car\PYZcb{}}\PY{l+s+s2}{ door and you have }\PY{l+s+s2}{\PYZob{}}\PY{l+s+s2}{\PYZsq{}}\PY{l+s+s2}{won}\PY{l+s+s2}{\PYZsq{}}\PY{l+s+s2}{ if car == [pick, notopen][not stay] else }\PY{l+s+s2}{\PYZsq{}}\PY{l+s+s2}{not won}\PY{l+s+s2}{\PYZsq{}}\PY{l+s+s2}{\PYZcb{} it.*}
        \PY{l+s+s2}{\PYZdq{}\PYZdq{}\PYZdq{}}\PY{p}{)}\PY{p}{)}
\end{Verbatim}


    \begin{center}\rule{0.5\linewidth}{\linethickness}\end{center}

\hypertarget{what-to-submit-to-the-course-page}{%
\subsection{What to submit to the course
page}\label{what-to-submit-to-the-course-page}}

Please save this notebook containing all of your work and submit it
using the link on the Moodle page. Please also run the following Python
code and copy and paste its output below it to the same link on the
Moodle page.

    \begin{Verbatim}[commandchars=\\\{\}]
{\color{incolor}In [{\color{incolor} }]:} \PY{n+nb}{print}\PY{p}{(}\PY{l+s+s2}{\PYZdq{}}\PY{l+s+s2}{Please copy and paste the following line into the textbox on the Moodle page:}\PY{l+s+s2}{\PYZdq{}}\PY{p}{)}
        \PY{n+nb}{print}\PY{p}{(}\PY{n}{f}\PY{l+s+s2}{\PYZdq{}}\PY{l+s+si}{\PYZob{}einstein\PYZcb{}}\PY{l+s+s2}{,}\PY{l+s+si}{\PYZob{}monty\PYZcb{}}\PY{l+s+s2}{,}\PY{l+s+si}{\PYZob{}savant\PYZcb{}}\PY{l+s+s2}{,}\PY{l+s+si}{\PYZob{}game\PYZcb{}}\PY{l+s+s2}{,}\PY{l+s+si}{\PYZob{}car\PYZcb{}}\PY{l+s+s2}{,}\PY{l+s+si}{\PYZob{}pick\PYZcb{}}\PY{l+s+s2}{,}\PY{l+s+si}{\PYZob{}stay\PYZcb{}}\PY{l+s+s2}{\PYZdq{}}\PY{p}{)}
\end{Verbatim}


    \hypertarget{end}{%
\subsection{End}\label{end}}


    % Add a bibliography block to the postdoc
    
    
    
    \end{document}
